\documentclass[final]{beamer}
%% Possible paper sizes: a0, a0b, a1, a2, a3, a4.
%% Possible orientations: portrait, landscape
%% Font sizes can be changed using the scale option.
\usepackage[size=a1,orientation=portrait,scale=1.1]{beamerposter}

\usetheme{gemini}
\usecolortheme{seagull}
\useinnertheme{rectangles}

% ====================
% Packages
% ====================

\usepackage[utf8]{inputenc}
\usepackage{graphicx}
\usepackage{booktabs}
\usepackage{tikz}
\usepackage{pgfplots}
\usepackage{multicol}

% ====================
% Lengths
% ====================

% If you have N columns, choose \sepwidth and \colwidth such that
% (N+1)*\sepwidth + N*\colwidth = \paperwidth
\newlength{\sepwidth}
\newlength{\colwidth}
\setlength{\sepwidth}{0.03\paperwidth}
\setlength{\colwidth}{0.45\paperwidth}

\newcommand{\separatorcolumn}{\begin{column}{\sepwidth}\end{column}}
\usepackage{caption}
\captionsetup[figure]{labelformat=empty} % Removes "Figure:" and the number
% ====================
% Logo (optional)
% ====================

% LaTeX logo taken from https://commons.wikimedia.org/wiki/File:LaTeX_logo.svg
% use this to include logos on the left and/or right side of the header:
\logoright{\includegraphics[height=3cm]{./logos/logo2}}
\logoleft{\includegraphics[height=3cm]{logos/IIST Logo.pdf}}

% ====================
% Footer (optional)
% ====================


% (can be left out to remove footer)

% ====================
% My own customization
% - BibLaTeX
% - Boxes with tcolorbox
% - User-defined commands
% ====================


%% Reference Sources
%\addbibresource{refs.bib}
\renewcommand{\pgfuseimage}[1]{\includegraphics[scale=2.0]{#1}}

\title{Meshfree Method for Stress Driven Beams}

\author{Akhil S L,  \and  I R Praveen Krishna }

\institute[IIST]{
	Department of Aerospace Engineering,\\
Indian Institute of Space Science and Technology, Thiruvananthapuram, Kerala }

\date{August 15, 2022}


\input{mydefs.tex}
\begin{document}
	
\begin{frame}[t]
	
	\begin{columns}[t]
	
	\begin{column}{2\colwidth+\sepwidth}
		\vspace{-0.5cm}
	\begin{block}{Introduction}
		\justifying
		Low-dimensional structures with dimensions in the micro-nano range exhibit size-dependent behavior that cannot be captured by local constitutive models. This deviation occurs because local models assume material point interactions are local, whereas size-dependent behavior arises from long-range interactions. While Eringen’s strain-driven nonlocal model has been widely used, it often results in ill-posed governing equations and paradoxical results for beam bending. The stress-driven nonlocal approach has emerged as a mathematically consistent and well-posed substitute. This research develops the Element-Free Galerkin (EFG) method for a stress-driven one-dimensional Bernoulli-Euler beam, utilizing its inherent nonlocal solution approximation to accurately simulate size effects.
	\end{block}
	\end{column}
	\end{columns}

	\begin{columns}[t]
		\separatorcolumn
		\begin{column}{\colwidth}
				\begin{defbox}{Scope}
					\begin{figure}
						\includegraphics[width = 0.82\textwidth]{pictures/applications}
						\caption{A MEMS resonator (\copyright Bhaskaran et al.), images from  \copyright scitime, a MEMS device by \copyright sensing-machines.}
					\end{figure}
				\end{defbox}
				
				\begin{defbox}{Theoretical Formulation and Methodology}{}
				Based on stress driven model, nonlocal elastic curvatures $\chi(x)$ are the output of a spatial convolution between a kernel function $\phi$ and the bending moment $M(x)$:
				\begin{equation}
					\chi(x)=\int_{0}^{L}\phi(x-\overline{x},c)\frac{M(x)}{EI}dx 
				\end{equation}
				\begin{itemize}
					\item [$\square$] Leads to a sixth order governing differential equation.
					\item [$\square$] EFG method rely of scattered nodes and moving least squares (MLS) approximations for constructing shape functions.
					\item [$\square$] A 6th-order spline weighting function is employed to provide the higher-order continuity required for approximating bending moments and shear forces.
				\end{itemize}	
				 
			\end{defbox}
			\begin{defbox}{Numerical Implementation}{}
				\begin{itemize}
					\item [$\square$] Essential and constitutive boundary conditions are enforced using a combination of Lagrange multipliers and scaled transformation.
					\item [$\square$] This methodology removes the requirement for constitutive continuity conditions found in element-based formulations.
					\item [$\square$] The method results in a fully populated stiffness matrix, requiring more computational effort than traditional FEM but it offers a more robust tool for nonlocal analysis.
				\end{itemize}
				

			\end{defbox}
				\begin{defbox}{Validation and Parametric Study}{}
				
					\begin{itemize}
					\item [$\square$] The SD-EFGM formulation was validated against experimental results from cantilever micro-beam deflection tests.
					\item [$\square$] For a beam height $h = 15 \mu m$, the model matches experimental data with a nonlocal parameter, $\lambda = 5.0$
					\item [$\square$] The model correctly predicts a reduction in nonlocal behavior as the beam length increases.
					\item[$\square$] Static results for simply supported, cantilever, and fixed-fixed beams under various loads align with analytical stress-driven benchmarks.
				\end{itemize}
				\begin{figure}
					\centering
					\includegraphics[width = 0.4\textwidth]{pictures/figure4}
					\vspace{-0.4 cm}
					\caption{Force-displacement relation, experimental and predicted}
					\label{figure2}
				\end{figure}   
			\end{defbox}
			\begin{defbox}{Effect of Nondimensional Length Scale}
				\begin{itemize}
					\item [$\square$] Accuracy is heavily dependent on the local support size, scaled by the nondimensional length scale parameter, $d_{max}$.
					\item [$\square$] Parametric studies show that error remains high if $d_{max}$ is below 50, with an optimal range for minimal error between 200 and 600.
					\item [$\square$] Stability requires a sufficiently large number of nodes; for example, at least 40 nodes are needed for moment balance in simply supported beams(length 25nm, width and height 1nm).
					\item [$\square$] Simulating nonlocal solids requires a support domain large enough to effectively include long range interactions.
				\end{itemize}
				\begin{figure}
					\centering
					\includegraphics[width = 0.4\textwidth]{pictures/figure5}
					\vspace{-0.4 cm}
					\caption{Error in displacement v./s Nondimensional length scale}
					\label{figure5}
				\end{figure}   
			\end{defbox}
		\end{column}
		
		\separatorcolumn
		
		\begin{column}{\colwidth}
			
				\begin{defbox}{Numerical Results}{}
				 
				\begin{figure}
					\centering
					\includegraphics[width = 0.82\textwidth]{pictures/cantileverBeam}
					\vspace{-0.4 cm}
					\caption{Normalized displacement, rotations, bending moment and shear force of cantilever beam}
					\label{fig:my_label}
				\end{figure}
			\end{defbox}
				\begin{defbox}{Numerical Results}{}
	
	\begin{figure}
		\centering
		\includegraphics[width = 0.82\textwidth]{pictures/ssb_combined}
		\vspace{-0.4 cm}
		\caption{Normalized displacement, rotations, bending moment and shear force of simply supported beam beam}
		\label{figssb}
	\end{figure}
\end{defbox}
			
			\vspace{0.2cm}
		\begin{defbox}{Conclusions}{}
		\begin{itemize}
			\item [$\square$] SD-EFGM provides an efficient numerical scheme for the static analysis of nonlocal Bernoulli-Euler beams.
			\item [$\square$] The displacement shows a consistent stiffening effect as nonlocal parameters increase.
			\item [$\square$] Accurate nonlocal simulation requires a non-compact support domain—with an optimal nondimensional length scale ($d_{max}$) between 200 and 600.
			\item [$\square$] The method accurately captures displacement, slopes, bending moments, and shear forces.
			\item [$\square$] Provides a robust alternative for element based formulation and offers scalability. 
			
		\end{itemize}
		\end{defbox}
					\begin{block}{References}
			
					% This prevents the "Entry Name" error and uses small text
					\setbeamertemplate{bibliography item}[text] 
					\renewcommand{\refname}{} % Removes the default "References" title inside the block
					\begin{footnotesize}
						\begin{thebibliography}{9}
							\setlength{\itemsep}{0.2ex} % Adjusts vertical spacing between refs
							
							\bibitem{akhil2025effect}
							S.~L. Akhil, I.~R. Praveen Krishna, and M.~Aswathy.
							\newblock Effect of non-dimensional length scale in element free Galerkin method for classical and strain driven nonlocal elasto-static problems.
							\newblock \textit{Computers \& Structures}, 312:107724, 2025.
							
							\bibitem{akhil2025element}
							S.~L. Akhil and I.~R. Praveen Krishna.
							\newblock Element-Free Galerkin Method for Elastostatic Analysis of Nonlocal Stress-Driven Bernoulli--Euler Beams.
							\newblock \textit{Journal of Engineering Mechanics}, 151(10):04025050, 2025.
							

							%\setcounter{enumiv}{2}
							\bibitem{patel2022novel}
							B.~N. Patel and S.~M. Srinivasan.
							\newblock Novel nickle foil micro-bend tests and the need for a relook at length scale parameter’s numerical value.
							\newblock \textit{Mechanics of Advanced Materials and Structures}, 29(25):3924--3933, 2022.
							
							\bibitem{romano2017nonlocal}
							G.~Romano and R.~Barretta.
							\newblock Nonlocal elasticity in nanobeams: the stress-driven integral model.
							\newblock \textit{International Journal of Engineering Science}, 115:14--27, 2017.
							
						\end{thebibliography}
					\end{footnotesize}
		
		\end{block}
		
		
		
		
		\end{column}		
		\separatorcolumn
	\end{columns}

% This fills the empty space and gives the overlay a stable base
\vfill 

% --- The Absolute Footer ---
\begin{tikzpicture}[remember picture, overlay]
	\node[
	anchor=south west, 
	inner sep=0pt, 
	outer sep=0pt,
	at={(current page.south west)} 
	] {
		\begin{tcolorbox}[
			enhanced,
			width=\paperwidth,
			sharp corners,
			boxrule=0pt,
			colback=myfooterblue,
			colframe=myfooterblue,
			colupper=white,
			boxsep=5mm, 
			top=0mm, bottom=0mm,
			left=2\sepwidth, right=2\sepwidth
			]
			\centering
			\large \textbf{Structural Dynamics and Vibration Laboratory} \hfill \textbf{IIST, Thiruvananthapuram, Kerala} \hfill \textbf{2026}
		\end{tcolorbox}
	};
\end{tikzpicture}

\end{frame}
\end{document}