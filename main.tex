\documentclass[final]{beamer}
%% Possible paper sizes: a0, a0b, a1, a2, a3, a4.
%% Possible orientations: portrait, landscape
%% Font sizes can be changed using the scale option.
\usepackage[size=a1,orientation=portrait,scale=1.1]{beamerposter}

\usetheme{gemini}
\usecolortheme{seagull}
\useinnertheme{rectangles}

% ====================
% Packages
% ====================

\usepackage[utf8]{inputenc}
\usepackage{graphicx}
\usepackage{booktabs}
\usepackage{tikz}
\usepackage{pgfplots}
\usepackage{multicol}

% ====================
% Lengths
% ====================

% If you have N columns, choose \sepwidth and \colwidth such that
% (N+1)*\sepwidth + N*\colwidth = \paperwidth
\newlength{\sepwidth}
\newlength{\colwidth}
\setlength{\sepwidth}{0.03\paperwidth}
\setlength{\colwidth}{0.45\paperwidth}

\newcommand{\separatorcolumn}{\begin{column}{\sepwidth}\end{column}}

% ====================
% Logo (optional)
% ====================

% LaTeX logo taken from https://commons.wikimedia.org/wiki/File:LaTeX_logo.svg
% use this to include logos on the left and/or right side of the header:
\logoright{\includegraphics[height=3cm]{./logos/logo2}}
\logoleft{\includegraphics[height=3cm]{logos/IIST Logo.pdf}}

% ====================
% Footer (optional)
% ====================


% (can be left out to remove footer)

% ====================
% My own customization
% - BibLaTeX
% - Boxes with tcolorbox
% - User-defined commands
% ====================


%% Reference Sources
%\addbibresource{refs.bib}
\renewcommand{\pgfuseimage}[1]{\includegraphics[scale=2.0]{#1}}

\title{Meshfree Method for Stress Driven Beams}

\author{Akhil S L  \and  I R Praveen Krishna }

\institute[IIST]{
	Department of Aerospace Engineering,\\
Indian Institute of Space Science and Technology, Thiruvananthapuram, Kerala }

\date{August 15, 2022}


\input{mydefs.tex}
\begin{document}
	
\begin{frame}[t]
	
	\begin{columns}[t]
	
	\begin{column}{2\colwidth+\sepwidth}
	\begin{block}{Introduction}
		\justifying
		Low-dimensional structures with dimensions in the micro-nano range exhibit size-dependent behavior that cannot be captured by local constitutive models. This deviation occurs because local models assume material point interactions are local, whereas size-dependent behavior arises from long-range interactions. While Eringen’s strain-driven nonlocal model has been widely used, it often results in ill-posed governing equations and paradoxical results for beam bending. The stress-driven nonlocal approach has emerged as a mathematically consistent and well-posed substitute. This research develops the Element-Free Galerkin (EFG) method for a stress-driven one-dimensional Bernoulli-Euler beam, utilizing its inherent nonlocal solution approximation to accurately simulate size effects.
	\end{block}
	\end{column}
	\end{columns}

	\begin{columns}[t]
		\separatorcolumn
		\begin{column}{\colwidth}

				\begin{defbox}{Normalized displacement of clamped-clamped beam}{}
				
				\begin{figure}
					\centering
					\includegraphics[width = 0.4\textwidth]{pictures/figure1}
					\vspace{-0.4 cm}
					\caption{Normalized displacemnt of cantilever beam}
					\label{fig:my_label}
				\end{figure}
				\vspace{-0.2 cm}
				\begin{figure}
					\centering
					\includegraphics[width = 0.4\textwidth]{pictures/figure2}
					\vspace{-0.4 cm}
					\caption{Domain and boundary conditions of the numerical model}
					\label{fig:my_label}
				\end{figure}   
			\end{defbox}
			\begin{defbox}{Normalized displacement of clamped-clamped beam}{}
				
				\begin{figure}
					\centering
					\includegraphics[width = 0.4\textwidth]{pictures/figure1}
					\vspace{-0.4 cm}
					\caption{Normalized displacemnt of cantilever beam}
					\label{fig:my_label}
				\end{figure}
				\vspace{-0.2 cm}
				\begin{figure}
					\centering
					\includegraphics[width = 0.4\textwidth]{pictures/figure2}
					\vspace{-0.4 cm}
					\caption{Domain and boundary conditions of the numerical model}
					\label{fig:my_label}
				\end{figure}   
			\end{defbox}
				\begin{defbox}{Normalized displacement of clamped-clamped beam}{}
				
				\begin{figure}
					\centering
					\includegraphics[width = 0.4\textwidth]{pictures/figure1}
					\vspace{-0.4 cm}
					\caption{Normalized displacemnt of cantilever beam}
					\label{fig:my_label}
				\end{figure}
				\vspace{-0.2 cm}
				\begin{figure}
					\centering
					\includegraphics[width = 0.4\textwidth]{pictures/figure2}
					\vspace{-0.4 cm}
					\caption{Domain and boundary conditions of the numerical model}
					\label{fig:my_label}
				\end{figure}   
			\end{defbox}
		\end{column}
		
		\separatorcolumn
		
		\begin{column}{\colwidth}
			
				\begin{defbox}{Normalized displacement of clamped-clamped beam}{}
				
				\begin{figure}
					\centering
					\includegraphics[width = 0.4\textwidth]{pictures/figure1}
					\vspace{-0.4 cm}
					\caption{Normalized displacemnt of cantilever beam}
					\label{fig:my_label}
				\end{figure}
				\vspace{-0.2 cm}
				\begin{figure}
					\centering
					\includegraphics[width = 0.4\textwidth]{pictures/figure2}
					\vspace{-0.4 cm}
					\caption{Domain and boundary conditions of the numerical model}
					\label{fig:my_label}
				\end{figure}   
			\end{defbox}
			
			\begin{defbox}{Normalized displacement of clamped-clamped beam}{}
			
			\begin{figure}
				\centering
				\includegraphics[width = 0.4\textwidth]{pictures/figure1}
				\vspace{-0.4 cm}
				\caption{Normalized displacemnt of cantilever beam}
				\label{fig:my_label}
			\end{figure}
			\vspace{-0.2 cm}
			\begin{figure}
				\centering
				\includegraphics[width = 0.4\textwidth]{pictures/figure2}
				\vspace{-0.4 cm}
				\caption{Domain and boundary conditions of the numerical model}
				\label{fig:my_label}
			\end{figure}   
		\end{defbox}
			
			
\begin{block}{References}
	% This prevents the "Entry Name" error and uses small text
	\setbeamertemplate{bibliography item}[text] 
	\renewcommand{\refname}{} % Removes the default "References" title inside the block
	
	\begin{footnotesize}
		\begin{thebibliography}{9}
			\setlength{\itemsep}{0.2ex} % Adjusts vertical spacing between refs
			
			\bibitem{akhil2022} 
			A. S. L. Akhil and I. R. P. Krishna, 
			\newblock "Stress Driven Beams using Meshfree Methods," 
			\newblock \textit{IIST Research Symposium}, 2022.
			
			\bibitem{eringen} 
			A. C. Eringen, 
			\newblock "Nonlocal Continuum Field Theories," 
			\newblock \textit{Springer}, 2002.
			
		\end{thebibliography}
	\end{footnotesize}
\end{block}
		\end{column}
		
		\separatorcolumn
	\end{columns}
% This fills the empty space and gives the overlay a stable base
\vfill 

% --- The Absolute Footer ---
\begin{tikzpicture}[remember picture, overlay]
	\node[
	anchor=south west, 
	inner sep=0pt, 
	outer sep=0pt,
	at={(current page.south west)} 
	] {
		\begin{tcolorbox}[
			enhanced,
			width=\paperwidth,
			sharp corners,
			boxrule=0pt,
			colback=myfooterblue,
			colframe=myfooterblue,
			colupper=white,
			boxsep=10mm, 
			top=0mm, bottom=0mm,
			left=2\sepwidth, right=2\sepwidth
			]
			\centering
			\large \textbf{Structural Dynamics and Vibration Laboratory} \hfill \textbf{IIST, Thiruvananthapuram} \hfill \textbf{2026}
		\end{tcolorbox}
	};
\end{tikzpicture}

\end{frame}
\end{document}